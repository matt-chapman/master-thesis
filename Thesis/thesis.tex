%%%%%%%%%%%%%%%%%%%%%%%%%%%%%%%%%% PREAMBLE %%%%%%%%%%%%%%%%%%%%%%%%%%%%%%%%%%%%

\documentclass{uvamscse}	% UvA thesis class

% numeric citations, name/title/year sorting
\usepackage[backend=biber, style=alphabetic, citestyle=alphabetic, sorting=nty]{biblatex}

% program listings environment, see uvamscse.cls
\input{program-listings}
\newcommand{\cmd}[1]{\texttt{$\backslash$#1}}

% todo notes, inc. some custom colours
\usepackage{xargs}                      % Use more than one optional parameter in a new commands
\usepackage[pdftex,dvipsnames]{xcolor}  % Coloured text etc.

\usepackage[colorinlistoftodos,prependcaption,textsize=small]{todonotes}
\newcommandx{\unsure}[2][1=]{\todo[linecolor=red,backgroundcolor=red!25,bordercolor=red,#1]{#2}}
\newcommandx{\change}[2][1=]{\todo[linecolor=blue,backgroundcolor=blue!25,bordercolor=blue,#1]{#2}}
\newcommandx{\info}[2][1=]{\todo[linecolor=OliveGreen,backgroundcolor=OliveGreen!25,bordercolor=OliveGreen,#1]{#2}}
\newcommandx{\improvement}[2][1=]{\todo[linecolor=Plum,backgroundcolor=Plum!25,bordercolor=Plum,#1]{#2}}
\newcommandx{\thiswillnotshow}[2][1=]{\todo[disable,#1]{#2}}

\title{Detecting Online Conversations Going Viral}
\coverpic[250pt]{figures/Buzzcapture_thunder.png}
\subtitle{Time-series aware evaluation of change detection algorithms}
\date{Spring 2017}

\author{Matt Chapman}
\authemail{matthew.chapman@student.uva.nl}
\host{Buzzcapture International, \url{http://www.buzzcapture.com}}
\supervisor{Evangelos Kanoulas, Universiteit van Amsterdam}

\abstract{
	\todo[inline]{Write abstract}
}

\bibliography{../bib/library.bib}

\begin{document}
\nocite{*}
\maketitle

%%%%%%%%%%%%%%%%%%%%%%%%%%%%%%%%%%%%%%%%%%%%%%%%%%%%%%%%%%%%%%%%%%%%%%%%%%%%%%%%

\chapter{Problem Statement \& Motivation}

\section{Problem Statement}

Within the domain of change detection algorithms there are a number of available methods for evaluating the efficiency
and accuracy of a given algorithm, depending on how the problem is framed. The different problem framings available are,
for example:\unsure{are these definitions correct?}

\begin{description}
	\item[Classification] Wherein the algorithm result fits into one of two (or more) classes, for example correct or
	incorrect.
	\item[Clustering] Wherein the algorithm results are formed into clusters and evaluated using clustering metrics.
	\item[Partitioning] Like clustering, but?\unsure{How is this different from clustering?}
	\item[Retrieval] Wherein the results of the algorithm are scored as a \emph{retrieval problem}, where detection
	relevance is taken into account, perhaps along with some form of temporal or redundancy penalty.
\end{description}

With this research, it is intended to address this dichotomy between approaches, and further suggest an evaluation
approach that is suitable for use when attempting to evaluate change detection algorithms applied to time-series data
specifically.

While change detection itself has been around since the 1930's (and \emph{online} change detection since the 1950's) it
is still a field that attracts new thoughts and approaches - one example of which being \citeauthor{Ginsberg2009}'s
\citetitle{Ginsberg2009} \cite{Ginsberg2009}.

Various attempts have been made to evaluate the myriad approaches to change detection (\cite{Buntain2014} for example)
but each of these attempts tend to frame the problem somewhat differently, or do not take into account change detection
in time-series data.

\improvement[inline]{this needs expansion, I think}

\section{Motivation}

Change detection first came about as a quality control measure in manufacturing, and methods within this domain are generally
referred to as \emph{control charts}. Since the inception of approaches such as CUSUM that provide the possibility for
on-line evaluation of continuous data streams, change detection has grown as a field. With applications such as epidemic
detection, online reputation management and infrastructure error detection, change detection is hugely useful.

This particular research is motivated specifically by the online reputation management sector. The business hosting this
research project (Buzzcapture International [\url{http://www.buzzcapture.com}]) is a Dutch online reputation management 
company that provides services to other businesses throughout europe. Chief among these is the BrandMonitor application,
 which, among other features, provides a rudimentary notification system for clients that is triggered once there is an
 increase in conversation volume of \(\%n\). It is the intention of this research to provide a robust evaluation method
 for change detection algorithms such that an approach that is most effective for this particular use case can be 
 selected and implemented.

\chapter{Research Method}

\section{Data Preparation}

\section{Scoring Metrics}

Example function:

\begin{equation}
	f(relevance, temporal\_penalty, redundancy\_penalty)
\end{equation}

\improvement[inline]{start designing scoring methods}

Possible relevance measure, where $t_0$ is the earliest a spike can be detected and $t_n$ is the time that the signal 
returns to normal. $f(x)$ describes the function of the curve:

\begin{equation}
	\int^{t_0}_{t_n} f(x) dx
\end{equation}
\info{planning to do something with the area under the curve for relevance}
\todo[inline]{start desigining relevance measure}

\chapter{Background \& Context}

\chapter{Research}

\chapter{Results}

\chapter{Analysis \& Conclusions}

%%%%%%%%%%%%%%%%%%%%%%%%%%%%%%%%%%%%%%%%%%%%%%%%%%%%%%%%%%%%%%%%%%%%%%%%%%%%%%%%

\printbibliography

\newpage

\listoftodos[ToDo Notes]

\end{document}
