\documentclass[../main.tex]{subfiles}

\begin{document}

The results of this thesis open up numerous avenues for future research in the field. The validity issues discussed in \autoref{threats} suggest that changes to the experimental method could result in more interesting results.

It would be interesting to formulate the `ideal metric' discussed in \autoref{conclusions} mathematically, and prove it's veracity through simulation studies and analysis of real-world data, much like that which was carried out in this thesis. This could result in novel and useful research in the field of change point detection, especially considering the results of the simulation studies carried out in this thesis. To be able to not only state that existing measures are deficient in some ways, but also be able to suggest an \emph{alternative} metric would be an excellent piece of additional work to carry out in the future.

Additional work should be carried out with variations on the calculation of the metrics discussed in this thesis. For example, this thesis operates under the assumption that anything other than an exact detection of a change point, matching the ground-truth for that data set is a failure case for the binary classification measures. It would be interesting and useful to carry out experiments that vary the `grace period' afforded to algorithm detections (say, allowing for a detection $n$ data points after the ground-truth to also be considered a correct detection) and allow for the detection of change points \emph{early} - that is, not penalising (or even perhaps rewarding) early detections. Research conducted in this manner would certainly be useful for the use-case of the host organisation for this thesis.

Further research should also be undertaken using much larger data sets, and more numerous instances of these data sets. One of the main threats to validity for this project is the small sample size of the real-world data corpus, and this is something that, if resolved, would result in a considerable amount of results for analysis. Further, with a large enough test corpus, it would be possible to conduct statistical significance testing between the results, to evaluate exactly how much measures disagree with each other, if at all.

The selection of penalty functions for real-world data would also be an interesting avenue for future research. It is not reasonable for a client of the host company to have knowledge of information criterion functions, or to be able to select one in an interface for using change point detection as a notification system. Thus, it would be interesting to carry out work on a corpus of real-world data sets to determine if there is an optimal penalty function or optimal fixed value penalty for data of the kind collected by the host organisation.

Further to this, there is also a possibility to experiment with other algorithm configuration options such as minimum segment length, or the maximum number of change points to search for. Experimenting with these values was done informally as part of the knowledge gathering process for this thesis, but it would be interesting to carry out formal research in this area to evaluate if mis-configuration of this values in situations where you could not intuitively `know' the correct values (in online data streaming situations, for example) would negatively affect the results, and to what degree.

While this thesis deals specifically with change point detection approaches, it would be valuable to compare the performance of these approaches with machine learning and natural language processing approaches. Analysis techniques such as sentiment analysis have been shown in the past to be good methods of detecting `events' occurring on social media \cite{Alvanaki2011}, and it may well be that those approaches are far more effective in this domain. The Text Retrieval Conference of 2016 carried out an evaluation of real-time summarisation approaches \cite{trec2016}, and supplied a number of scoring methods which were used as part of this evaluation. It would be useful to conduct a replication study utilising a corpus of data taken from the host company of this thesis and apply the approaches submitted to that track to it. In that way, the effectiveness of content summarisation studies could be evaluated as a method for detecting anomalous changes in subject, rather than just conversation volume.

\end{document}