\documentclass[../main.tex]{subfiles}

\begin{document}

The results of this thesis open up numerous avenues for future research in the field. The validity issues discussed in \autoref{threats} suggest that changes to the experimental method could result in more interesting results.

It would also be interesting to formulate the `ideal metric' discussed in \autoref{ideal metric} mathematically, and prove it's veracity through simulation studies and analysis of real world data, much like that which was carried out in this thesis.

Further research should also be undertaken using much larger data sets, and more numerous instances of these data sets. One of the main threats to validity for this project is the small sample size of the real world data corpus, and this is something that, if resolved, would result in a considerable amount of results for analysis. Further, with a large enough test corpus, it would be possible to conduct statistical significance testing between the results, to evaluate exactly how much measures disagree with each other, if at all.

While this thesis deals specifically with change point detection approaches, it would be valuable to compare the performance of these approaches with machine learning and natural language processing approaches. Analysis techniques such as sentiment analysis have been shown in the past to be good methods of detecting `events' occurring on social media, and it may well be that those approaches are far more effective in this domain. The Text Retrieval Conference of 2016 carried out an evaluation of real-time summarisation approaches \cite{trec2016}, and supplied a number of scoring methods which were used as part of this evaluation. It would be useful to conduct a replication study utilising a corpus of data taken from the host company of this thesis and apply the approaches submitted to that track to it. In that way, the effectiveness of content summarisation studies could be evaluated as a method for detecting anomalous changes in subject, rather than just conversation volume.

\end{document}