\documentclass[../main.tex]{subfiles}

\begin{document}

What follows is an annotated bibliography of works that were read as part of the preparation for this research. It is not an exhaustive list, merely a set of papers, books and other publications which are relevant to this research or provided some insight into how best to carry it out. Full bibliographical citations are provided for each text.\bigskip

\acite{Buntain2014}{

One of the first papers on the subject read by the thesis author, this paper discusses and compares three different types of change detection algorithm: the \textit{Likelihood Ratio Test} (LRT), the \textit{Cumulative Sum} (CUSUM) test, and the \textit{Kernel-based Change Detection} (KCD) algorithm. The authors of this paper also kindly provided source code implementations of these algorithms, once contact was established. The algorithms were applied to several data sets, such as historical Bitcoin valuations and data from structural stress sensors.\bigskip}

\acite{Kawahara2009}{

This paper proposes what the authors call a ``novel non-parametric change detection algorithm''\cite{Kawahara2009}. It is applied alongside four other approaches, against three different data sets generated by models borrowed from other research papers on time-series data change detection. The approaches are evaluated according to accuracy rate and degree, though not for performance or other relevant metrics.\bigskip}

\acite{Kifer2004}{

This paper, much like the one written by Kawahara and Sugiyama, presents a novel change detection algorithm for evaluation. However, this particular paper differs in that the presented algorithm is also useful for estimation of the detected change. They also discuss the use of a `two window' approach to applying change detection algorithms to data streams, so as to limit memory usage in practice. This part in particular is relevant to the business case of the organisation hosting this thesis.\bigskip}

\acite{Madrid2004}{

This paper served as the basis of two of the algorithms tested in the paper written by \citeauthor{Buntain2014} This paper also provides it’s own evaluations of the approaches contained therein.\bigskip}

\acite{Desobry2005}{

Desobry, Davy, and Doncarli write concerning an implementation of a Kernel Change Detection (KCD) algorithm that is considered \textit{online} - that is, applicable to moving and updating data sets such as those that are processed by the host organisation of this thesis. This is particularly important, as any final implementation of a change detection method in the production systems of the thesis host company will be handling online data.

This paper makes use of Receiver Operating Characteristic curves to show the accuracy of change point detection algorithms, which is a variation on standard binary classification measures concerning false/true positives/negatives.\bigskip}

\acite{Tartakovsky2006}{

This paper discusses applying a CUSUM (Cumulative Sum) approach for detection of network intrusions. The approach taken by Tartakovsky et al. was intended to research the possibility of change detection while maintaining a low rate of false alarms. The idea was to apply an algorithm for this purpose that would not need to take into account pre-change and post-change models in order to be effective.\bigskip}

\acite{Tartakovsky2005}{

A collection of papers and discussions by the same authors as the above papers, expanding somewhat on the problem they tackled and the solutions found.\bigskip}

\acite{Matteson2012}{

Discussion of an `offline' (that is, applying an algorithm to a fixed data-set as opposed to processing moving `live' data) approach to change detection in multi-variate data. The authors carry out a simulation study to compare various approaches to this problem and present their results.\bigskip}

\acite{Siegmund1995}{

A paper on using a \textit{Generalized Likelihood Ratio} test for the detection of change points in a data set. An old piece of text that pre-dates social media, yet is still useful in explaining how the GLR test can be used for the detection of change points. The GLR approach is compared against standard CUSUM tests.\bigskip}

\acite{Willsky1976}{

Much like the paper written by Siegmund and Venkatraman, this paper pre-dates social media as a data source to which change detection algorithms could be applied. It is however, a useful look into how change detection algorithms have progressed and improved over the years. This paper is primarily focussed on uses of change detection in automated controls (being that it was published in the IEEE Transactions on Automatic Control), but is still a useful resource to understand how change detection algorithms can be evaluated.\bigskip}

\acite{Lai1999}{

Another paper concerning change point detection in control systems, Lai and Shan write primarily regarding Generalised Likelihood Ratio tests, and how these can be ‘configured’, specifically with regards to window size.\bigskip}

\acite{Bersimis2007}{

\noindent Control charts are a mechanism used in various industries to decide whether a given process is `in control' or `out of control'. In this way, control charts are used to discover outlying or anomalous results in a given data set, or inform operators of a sudden change in the data. This makes control charts particularly relevant to my research.

This particular paper discusses various approaches to control charts, and the methods behind their operation.\bigskip}

\acite{Downey2008}{
Downey discusses his creation of a `novel change detection algorithm' that can also be used for ``\dots predicting the distribution of the next point in the series.''\cite{downey2008novel}

He discusses in some detail the difference between online and offline change detection algorithms, and compares his implementation of a new algorithm with implementations of existing and established algorithms.}

\acite{Killick2011}{
    This paper carries out research similar in some ways to that being conducted in this thesis. Change detection methods are briefly explained and then applied to time series data from an oceanographic study. The ability for change detection approaches to correctly identify the onset and end of `storm seasons' - which form the ground truth in this study.
}

\acite{Matteson2012}{
    A comparison study of several change point detection methods. This study utilises the Rand Index as a measure for comparing algorithm output against a ground truth, and the Adjusted Rand Index for computing the similarity between two computed outputs. The approaches are then applied to various real world data sets from fields such as genetics and finance. This paper formed the basis for the decision to concentrate mainly on the performance of clustering metrics in the field of change point detection evaluation.
}

\acite{Killick2011a}{
    The paper that initially proposed the PELT algorithm for change point detection. The paper mainly concentrates on the computational cost of the PELT algorithm, comparing to other, existing methods of change point computation. Some analysis on the basis of algorithm accuracy is carried out, though limited to the use of binary classification measures.
}

\acite{Pelecanos2010}{
    A paper written concerning possible approaches for the detection of disease outbreaks. This publication was not interesting in terms of the approaches used, but rather in how they were evaluated. This paper restricts itself to the use of binary classification measures, tallying true/false positives and true/false negatives.
}

\acite{Qahtan2015}{
    An evaluation of change point detection methods, using traditional binary classification measures. The measure results are presented purely in the form of true positives, late detections, false positives and false negatives, and thus are easily digested and if necessary translated into the usual terms of recall, precision and F1.
}

\acite{Amigo2009}{
    A detailed evaluation of BCubed as a measure for document clustering. It was based off of this paper that BCubed was chosen as an additional metric to test. BCubed has a good reputation for accuracy, and this paper backs up the ability for BCubed to be effective with detailed comparisons of it's constraints and ability to perform correctly in various situations.

}

\acite{Alvanaki2011}{
    An alternative method of achieving the business goals of the host organisation of this thesis, \citetitle{Alvanaki2011} provides an explanation and example implementation of a method for detecting changes in \emph{conversation topic} as opposed to conversation volume.
}

\acite{Basseville1993}{
    The seminal text on change point detection methods and evaluation. This text is considerably old at the time of writing, but is still an invaluable source of insights and explanations as to how change point detection works, and the criteria by which they should be evaluated.
}

\acite{Dasu2009}{
    An interesting text discussing the application of distributional shift detection as a change detection measure. The approach is not necessary applicable to the work being carried out in this thesis, but it is interesting to read nonetheless - considering the differences between this approach and those tested in this research.
}

\acite{Ginsberg2009}{
    Another study in the same vein as \cite{Pelecanos2010}, this time carried out by Google. This study uses data streams to attempt to predict influenza outbreaks with surprising effectiveness. Here, a log-likelihood approach is used to determine the probability that a random visit by a doctor is related to an illness like influenza.
}

\acite{Kulldorff2005}{
    An additional study in the same category as \cite{Ginsberg2009} and \cite{Pelecanos2010}, it is interesting to this project due to it's use of observed/expected cases to prove the veracity of their prediction model, in the same manner as binary classification scores are utilised in other publications.
}

\acite{Tran2014}{
    A paper discussing the importance of change point detection techniques when applied to \emph{big data} scenarios. No comparison study is carried out, but the paper does serve to espouse the importance of change point detection in scenarios where the sheer \emph{volume} of data may serve to make many approaches impractical.
}

\acite{Xu2011}{
    This study compares various approaches for the application of event detection in data streams. It provides experiments evaluating approaches for topic change detection in documents and news streams, and also event detection in surveillance video streams. In a similar fashion to \cite{Alvanaki2011}, it focuses mainly on using analysis techniques to spot changes in topic in documents as opposed to simply measuring volume. 
}

\end{document}