\documentclass[a4paper]{scrartcl}
\usepackage[utf8]{inputenc}
\usepackage[sorting=nty,style=numeric,backend=biber]{biblatex}
\usepackage{courier}
\usepackage[hidelinks]{hyperref}
\usepackage{xcolor}
\usepackage{calc}
\hypersetup{
    colorlinks,
    linkcolor={red!50!black},
    citecolor={blue!80!black},
    urlcolor={blue!80!black}
}
\usepackage{multicol}

\newcommand{\acite}[2]{
\noindent\cite{#1}~\parbox[t]{\linewidth-17.8pt}{\vspace{-.65em}\textbf{\fullcite{#1}}}\begin{quote}#2\end{quote}
}

\title{Weekly Progress Report}
\date{Week 14, 03/04 - 07/04}
\author{Matt Chapman}
\bibliography{../../bib/library.bib}

\begin{document}
\maketitle

\section{Dataset Selection}

Selecting the correct dataset to carry out my analyses is one of the most important parts of the work I will carry out in this project. To this end, I will explain the data set that I will be working with, as well as my rationale for selecting it.

I have chosen to work exclusively with Dutch language Twitter data for the following reasons:

\begin{itemize}
	\item The vast majority of Buzzcapture clients are Dutch companies, so it is more likely that I will be able to find relevant data by restricting gathered conversations by language.
	\item Twitter is \emph{the} platform for breaking news and discussions thereof. It is likely that changes in volume that the likes of which I am looking for, will happen on Twitter before they happen on other platforms such as Facebook, Pinterest or Facebook.
\end{itemize}

\newpage
\printbibliography

\end{document}