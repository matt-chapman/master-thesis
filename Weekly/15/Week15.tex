\documentclass{mattreport}

\title{Weekly Progress Report}
\subtitle{Detecting Conversations Going Viral}
\date{Week 15, 10/04 - 14/04}
\author{Matt Chapman}
\bibliography{../../bib/library.bib}

\begin{document}
\maketitle

\section{General Notes}

Not the most productive of weeks. Time lost on Monday - Thursday due to university and personal commitments. However, progress has been made in the following areas:

\begin{itemize}
    \item Generating test data is reasonably simple - the only issue of concern here is time to complete the export.
    \item Consensus at the moment (within the team) is that 2 months of data, limited to Dutch language, will provide the most useful data.
    \item Data Export format is CSV, easily customised read and processed using the Python \texttt{Pandas} package.
    \item I have formulated some interesting queries that filter the data and produce some human readable spikes that can be manually annotated.
    \item An epic and stories within it has been created in the Buzzcapture JIRA instance, to allow for sprint planning. Sprints at Buzzcapture are (at the moment) 2 weeks in length.
    \item I have started building a Mendeley bibliography to manage sources and citations through the project.
    \item Regarding the bibliography, I have added some more useful sources from a couple of the suggested Special Interest Groups. This will continue through the project.
\end{itemize}

\section{Following Week}

The following tasks are due to be completed in the next week:

\begin{itemize}
    \item Finalisation of dataset upon agreement with supervisor
    \item Initial processing of dataset to filter on agreed terms
    \item Plotting and annotation of data set
    \item Build familiarity with Python data science tools. Feedback on this appreciated.
\end{itemize}

In addition to these tasks, the following issues need to be addressed regarding the project in general:

\begin{itemize}
    \item Research questions need solidifying. I am not currently happy with the generality of the research questions I have, and would like to refine these into concrete subquestions.
    \item The hypothesis I have does not feel suitable. I would also like to refine this further.
\end{itemize}

\newpage
\printbibliography

\end{document}
